%!TEX TS-program = xelatex
%!TEX encoding = UTF-8 Unicode
\documentclass[11pt,a4paper]{moderncv}
% moderncv themes
%\moderncvtheme[blue]{casual}                 % optional argument are 'blue' (default), 'orange', 'red', 'green', 'grey' and 'roman' (for roman fonts, instead of sans serif fonts)
\moderncvtheme[blue]{classic}                % idem
\usepackage{xunicode, xltxtra}
%\usepackage[colorlinks,linkcolor=red]{hyperref}
\XeTeXlinebreaklocale "zh"
\widowpenalty=10000

%\setmainfont[Mapping=tex-text]{文泉驿正黑}

% character encoding
\usepackage[utf8]{inputenc}                   % replace by the encoding you are using
\usepackage{CJK}

% adjust the page margins
\usepackage[scale=0.9]{geometry}
\recomputelengths                             % required when changes are made to page layout lengths
\setmainfont[Mapping=tex-text]{Hiragino Sans GB}
\setsansfont[Mapping=tex-text]{Hiragino Sans GB}
\CJKtilde

% personal data

%% start of file `template-zh.tex'.
%% Copyright 2006-2012 Xavier Danaux (xdanaux@gmail.com).
%
% This work may be distributed and/or modified under the
% conditions of the LaTeX Project Public License version 1.3c,
% available at http://www.latex-project.org/lppl/.

% 个人信息
\firstname{张}
\familyname{曼妮}
\title{个人简历}                      % 可选项、如不需要可删除本行
\address{住址:北京市海淀区西二旗铭科苑}{籍贯:湖北\,\,年龄:29\,\,}             % 可选项、如不需要可删除本行
\mobile{+86~18217376283}                         % 可选项、如不需要可删除本行
%\phone{+2~(345)~678~901}                          % 可选项、如不需要可删除本行
%\fax{+3~(456)~789~012}                            % 可选项、如不需要可删除本行
\email{footloosemandy@gmail.com}                    % 可选项、如不需要可删除本行
%\homepage{dinever.com}                  % 可选项、如不需要可删除本行
%\photo[64pt]{421281199009120017.jpg}                  % ‘64pt’是图片必须压缩至的高度、‘0.4pt‘是图片边框的宽度 (如不需要可调节至0pt)、’picture‘ 是图片文件的名字;可选项、如不需要可删除本行

\begin{document}
\maketitle
\renewcommand{\baselinestretch}{1}

\section{教育背景}
\cventry{2014 -- 2017}{硕士}{华东师范大学}{软件工程}{}{}  % 第3到第6编码可留白
\cventry{2008 -- 2012}{学士}{华中师范大学}{通信工程}{}{}
%\cventry{2005 -- 2008}{高中}{湖北省鄂南高级中学}{}{}{}  % 第3到第6编码可留白

%\section{毕业论文}
%\cvitem{题目}{\emph{题目}}
%\cvitem{导师}{导师}
%\cvitem{说明}{\small 论文简介}

\section{项目经历}
\renewcommand{\baselinestretch}{1}
\cventry{2019.1--至今}{北京翰海星云科技有限公司}{}{}{}
{Clustar\,\,AI, 面向企业的在线机器学习平台, 它采用RDMA网络,具有优越的分布式模型训练性能。
\begin{itemize}
\item 参与TensorFlow通信模块GDR的优化工作,利用RDMA技术传输远程GPU内存中的Tensor,绕过主机内存和CPU,提升Tensorflow框架的网络传输性能。
\item 优化Tensorflow模型训练的分布式策略,提升大规模集群下模型的训练速度。
\end{itemize}}
\\
\cventry{2018.5\\--2019.1}{思科中国研发中心}{}{}{}
{Cloud\,\,CMTS,基于微服务的下一代PE路由。采用k8s/docker架构部署,实现嵌入式设备向cloud的转换。
\begin{itemize}
\item 负责Admission control相关的microservice开发,基于当前可用带宽拒绝/接受新的traffic flow请求。避免当过多流共享同一个链路时,导致所有流的严重退化。提供CIR service flow,over DOCSIS (VDOC),Multicast QoS的相关支持
\end{itemize}}
\\
\cventry{2017.7\\--2019.1}{思科中国研发中心}{}{}{}
{Cisco cBR系列接入网融合宽带路由,作为单一设备连接调制解调器cable modem或数字机顶盒STB向住宅和商业用户提供高速数据、宽带和IP电话服务。\\
个人职责:
\begin{itemize}
\item 负责High\,\,Availability相关功能模块的开发和维护,提供备卡1:N的切换保护,保证服务的高可靠性
\item 解决HA相关问题,在线处理全球用户case
\item 参与cBR8新一代产品中Remote RHY设备上HA功能的开发
\end{itemize}}

\cventry{2012--2014}{上海斐讯数据通信公司}{}{}{}
{参与多款EPON,GPON,无线家庭网关的设备驱动开发涉及sdr,ddr,flash,usb,i2c,了解应用层的基本协议,
熟悉LINUX BSP,掌握一般软件测试方法}

\section{科研经历}\cvline{}{基于ROS机器人操作系统制作一款低成本室内自主移动社交机器人}
\cvline{}{发表论文: \href{https://ieeexplore.ieee.org/abstract/document/7558699}{\textcolor[rgb]{0,0,0}{Formally verifying navigation safety for ground robots(In 2016 IEEE International Conference on Mechatronics and Automation.1000--1005)}}}

\section{语言技能}
\cvline{英语}{CET-6,可作为工作语言,可以读写开发文档,看论文}

\section{专业技能}
\renewcommand{\baselinestretch}{1}

\cvline{}{熟练掌握c/go/python开发, 熟悉常见算法和数据结构}
\cvline{}{熟悉RDMA,包括RoCE和IB,了解TensorFlow网络架构}
\cvline{}{掌握kubernets框架,熟悉docker开发}
\cvline{}{熟悉linux操作系统,shell脚本}
\cvline{}{掌握kafka,zmq等messgae bus,熟悉cassandra,mongodb,etcd等存储服务}
\cvline{}{掌握git/svn/acme等软件版本管理工具,熟悉CI/CD工具Jenkins}





%\cvline{意大利语}{\textbf{QCER A2},具有基本的日常听说能力}


\end{document}


%% 文件结尾 `template-zh.tex'.
